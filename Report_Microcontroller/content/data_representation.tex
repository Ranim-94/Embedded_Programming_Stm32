

\chapter{Data Representation}

\section{Introduction}

The goal of this chapter is review and explain some common programming concept used in embedded context, such as:

	\begin{itemize}

	\item Data representation: binary, hexadecimal,$\cdots$
	
	\begin{itemize}
	\item Construct memory map table of hardware peripherals
	\end{itemize}

	\item Bitwise operation
	
		\begin{itemize}
		\item To manipulate registers (set or clear some bits)
		\end{itemize}			
	
	\end{itemize}

\section{Source}

\begin{itemize}

\item chapter 1 and 2 from \cite{book_Embedded_systems_ARM_Cortex_M_YifengZhu},  

\end{itemize}

\newpage
\section{Bit, Byte Word}

\begin{itemize}

\item Computers handle info in terms of \tbi{group of bits} $\rightarrow$ the idea of \tbi{byte} $\leftrightarrow$ a grouping of 8 bits

\item There are also othre grouping in terms of \tbi{multiple of bytes} $\leftrightarrow$  \tbi{larger unit called word}

	\begin{itemize}
	\item The word size is machine dependent $\leftrightarrow$ on a x-86 archi word = 64 bits = 8 byetes, whereas arm-cortex M archi where word = 32 bits = 4 bytes
	\end{itemize}

\item Memory of computers and processor are \tbi{byte addressable} $\leftrightarrow$ every byte in the memory has its unique addresse

	\begin{itemize}
	\item Example in a micrcontroller: in stm32f407 discovery board, each peripheral registers has its unique addresse
	\end{itemize}

\end{itemize}

\section{Data Types in C}

\todo{Data types in C}

\begin{itemize}

\item \textit{Read section 2.1 from \cite{book_Embedded_systems_ARM_Cortex_M_YifengZhu}, and try with other books}

\item \textit{To insert a table about the range of values}

\item  \textit{Maybe see if there are some codes about limit value, min and max}

\end{itemize}


\section{Number System and Bases}

see section 2.2 in \cite{book_Embedded_systems_ARM_Cortex_M_YifengZhu}

\begin{itemize}

\item Binary values are represtented in base 16 in modern computers

\item Convert from base 2 $\rightarrow$ base 16: separate each 4 bits and take the equivalent hex digit, as shown in the conversion table

\end{itemize}

\todo{conversion table} \textit{Insert the conversion table later}

\subsection{Base Representation in C code}

\begin{itemize}

\item There is what we call \tbi{suffixes} $\rightarrow$ force the compiler to treat a constant as an explicitly specified data type

\item Used suffixes in embedded programming:

	\begin{itemize}
	\item \verb|0x| for base 16. 
	
	Example: \verb|uint p = 0xFFFF|
	
	
	\item \verb|UL|:  \verb|U| for unisigned and \verb|L| for long int.
	
	Example: \verb|0xFFFFUL|
	
	\end{itemize}


\end{itemize}


%============ END Chapter ============
